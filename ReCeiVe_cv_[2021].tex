%%%%%%%%%%%%%%%%%%%%%%%%%%%%%%%%%%%%%%%%%
% ReCeiVe
% LuaLaTeX Template
% Version 1.12.0 (10/10/2020)
%
% Original authors:
% Ged Lex (gedlex@hotmail.ch)
%
% Important note:
% This template must be compiled with LuaLaTeX, the below lines will ensure this
%!TEX TS-program = lualatex
%!TEX encoding = UTF-8 Unicode
%%%%%%%%%%%%%%%%%%%%%%%%%%%%%%%%%%%%%%%%%

%----------------------------------------------------------------------------------------
%	PACKAGES AND OTHER DOCUMENT CONFIGURATIONS
%----------------------------------------------------------------------------------------
\documentclass[rightPos]{ReCeiVe}      % By default, use 'letterpaper' for US letter
%\geometry{letterpaper, % paper size
%          left=1.75cm, right=1.75cm, top=1.5cm, bottom=1.5cm, footskip=.6cm, headsep=.5cm, % margins
%          showframe}                  % show geometry frame
%          heightrounded}              % avoid underfull vbox warning
          
% Color for highlights
%\definecolor{highlight}{HTML}{EB9534} % Specify your own color
%\colorlet{highlight}{white}           % Set predefined color
% Default colors include: darkgray, gray, lightgray, lightblue, orange, red, concrete

% Colors for text - uncomment and modify
%\definecolor{darktext}{HTML}{414141}
%\definecolor{text}{HTML}{414141}
%\definecolor{graytext}{HTML}{414141}
%\definecolor{lighttext}{HTML}{414141}

%----------------------------------------------------------------------------------------
%	PERSONAL INFORMATION
%	Comment any of the lines below if they are not required
%----------------------------------------------------------------------------------------
\background{pics/background5.pdf}
% \photo[circle,edge,fill,left]{2.5cm}{pics/pp2.pdf}
\name{Borong}{Xu}
% \address{Turicum, Switzerland}
\mobile{+852 8491 8005}
\email{borongxu@outlook.com}
\github{boring180}
\linkedin{borong-xu-52829a293}
\homepage{boring180.github.io/}
%\twitter{@twit}
%\xing{xing name}
%\stackoverflow{SOid}{SOname}
%\skype{skypeid}
%\reddit{reddit account}
%\extrainfo{info}

\position{BEng in Computer Engineering} % Your expertise/fields
\headwords{Computer Engineering Student, STEM educator} % A few headwords to describe yourself
% \quote{"Some men see things as they are, and ask why. I dream of things that never were, and ask why not." - Robert Kennedy} % A quote or statement

%----------------------------------------------------------------------------------------
%	SIDEBAR CONTENT
%	Fill in the information you would like to add to your sidebar
%----------------------------------------------------------------------------------------

\aboutMe{Engineer, software geek, technical enthusiast and carpenter}
\userSection[Skills]{
\begin{itemize}
    \item C/C++
    \item Python
    \item ROS Noetic/Foxy/Humble
\end{itemize}
}
% \skillset[title/Hi]{MATLAB/.,Python/.4,VBA/.6,Java/.2,C++/.4}
\languages{{English/Fluent (IELTS 7.5)},{Mandarin/Mother tongue}}
% \hobbies{Lorem ipsum dolor sit amet, consectetur adipiscing elit}
% \nonProfit{%
% 	\begin{itemize}
% 	\item{Lorem ipsum dolor sit amet}
% 	\item{Lorem ipsum dolor sit amet, consectetur adipiscing elit. Donec finibus eu felis ullamcorper finibus.}
% 	\end{itemize}
% }
% Usage: \userSection[<section title>]{<content>}

%----------------------------------------------------------------------------------------
\begin{document}
% Print the header
% Usage: \makecvheader[<position>]
\makecvheader[L]
% Print the sidebar
% Usage: \makecvsidebar[xOffset/value,yOffset/value,noRadius|radius]
\makecvsidebar

%----------------------------------------------------------------------------------------
%	CV/RESUME CONTENT
%	Each section is imported separately, open each file in order to modify it
%----------------------------------------------------------------------------------------
% Remove space before first section
\vspace{-\acvSectionTopSkip}
\section{Education}
\cventry
{Bachelor of Engineering in Computer Engineering} % Education Title
{The Hong Kong University of Science and Technology (HKUST)} % Organization
{Hong Kong SAR} % Location
{Expected Graduation: June 2026} % Date
\begin{cvitems}
    \item {GPA 3.8/4.3 (Top 4\% among 150 students)}
    \item {Activities:}
        \begin{itemize}
            \begin{multicols}{2}
                \item Student Representative
                \item Student Advisor
            \end{multicols}
        \end{itemize}
    \item {Course highlights:}
        \begin{itemize}
            \begin{multicols}{2}
                \item Probability and Statistics
                \item Algorithms
                \item Mobile Robotics
                \item Deep Learning
                \item Computer Vision
                \item Machine Learning
            \end{multicols}
        \end{itemize}
\end{cvitems}

\cventry
{High School Diploma(GaoKao)} % Education Title
{High School Affiliated to Renmin University of China (RDFZ)} % Organization
{Beijing, China} % Location
{Graduated June 2022} % Date
\begin{cvitems}
\item {Activities:}
\begin{itemize}
    \item First Robotics Club, Chief Engineer (season 2021)
    \item First Robotics Club, Alumni Mentor (season 2022-2024)
\end{itemize}
\end{cvitems}
\section{Experience}
% \cventry
% {Stable Diffusion Image Generation Model Fine Tuning} % Job Title
% {Undergraduate Research Opportunity(UROP)} % Organization
% {HKUST} % Location
% {2024.2 - 2024.5} % Date
% \begin{cvitems}
% \item {Fine-tuned a Stable Diffusion image generation model using LoRA, achieving improved style adaptation. Further enhanced generation stability and controllability by integrating ControlNet for precise conditioning.}
% \end{cvitems}

\cventry
{LLM Agent Development} % Job Title
{Undergraduate Research Opportunity (UROP)} % Organization
{Supervised by Prof. Shenghui Song} % Location
{Jun 2024 - Present} % Date
\begin{cvitems}
\item {Designed and implemented a Large Language Model (LLM) agent capable of playing board games. Conducted comprehensive evaluations of agent strategies and performance.}
\end{cvitems}

\cventry
{Mobile Robot Algorithm Implementation} % Job Title
{Course Project} % Organization
{Supervised by Prof. Shaojie Shen} % Location
{Sep 2024 - Dec 2024} % Date
\begin{cvitems}
\item {Developed mobile robot algorithms from scratch, including ICP Odometry, EKF SLAM, and A* path planning within a simulation environment, enabling robust localization, mapping, and navigation.}
\end{cvitems}

\cventry
{Quadruped Robot Control} % Job Title
{Research Practicum} % Organization
{Supervised by Prof. Maurice Pagnucco and Prof. Yang Song} % Location
{Feb 2025 - Aug 2025} % Date
\begin{cvitems}
\item {Built and programmed a Unitree GO2 quadruped robot to perform complex tasks, integrating depth camera for object localization and LiDAR for SLAM.}
\end{cvitems}

\cventry
{Underwater Multi-Camera Localization System} % Job Title
{Final Year Project} % Organization
{Supervised by Prof. Huan Yin} % Location
{Jun 2025 - Present} % Date
\begin{cvitems}
\item {Developed a Multi-Camera Vision-Based Localization System for Tracking Multiple Autonomous Underwater Vehicles (AUVs).}
\end{cvitems}
\section{Honors}
\begin{cvhonors}

\cvhonor
{1st Place} % Position
{Circular Economy Challenge} % Event
{Turicum, CH} % Location
{2021} % Date

\end{cvhonors}

%----------------------------------------------------------------------------------------
% Usage: \makecvfooter(<left>}{<center>}{<right>)
\makecvfooter{\today}{Ged Lex~~~·~~~CV}{\LaTeX{}}
\end{document}

